\documentclass{article}
\usepackage{graphicx} % Required for inserting images

\title{Algebra: Chapter Zero Solutions}
\author{jacobszelko }
\date{May 2024}

\begin{document}

\maketitle

\section{Introduction}

#=

Algebra: Chapter Zero Chapter 2, Exercise 2.7

A quick computational approach to get a feel for what the approach underlying this solution may be.
Based on conversation, seems like the identity and "flip" of the combination is the key here with p1 being identity and p3 being the "flip".

=#

p1 = [1, 2, 3, 4]
p2 = [4, 1, 2, 3]
p3 = [3, 4, 1, 2]
p4 = [2, 3, 4, 1]
p5 = [2, 1, 4, 3]
p6 = [4, 3, 2, 1]
p7 = [3, 2, 1, 4]
p8 = [1, 4, 3, 2]

perms = [p1, p2, p3, p4, p5, p6, p7, p8]

labels = [
    "p1",
    "p2",
    "p3",
    "p4",
    "p5",
    "p6",
    "p7",
    "p8"
]

perm_labels = Iterators.product(labels, labels) |> collect |> x -> reshape(x, (1, 64))

commuting = []
for (idx, val) in enumerate(Iterators.product(perms, perms) |> collect |> x -> reshape(x, (1, 64)))
    a = val[1]
    b = val[2]

    c1 = []
    c2 = []

    for i in 1:4
        push!(c1, b[a[i]])
        push!(c2, a[b[i]])
    end
    if c1 == c2
        push!(commuting, perm_labels[idx])
    end
end




\end{document}
